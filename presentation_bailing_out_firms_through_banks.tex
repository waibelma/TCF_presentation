%\documentclass[handout]{beamer} % Activate only if handout slides of presentation desired.
\documentclass[13.8pt]{beamer}

\usetheme{CambridgeUS}

% Packages
\usepackage[utf8]{inputenc}
\usepackage[T1]{fontenc}
\usepackage{amsmath}
\usepackage{graphicx}
\usepackage{eurosym}
\usepackage{hyperref}
\usepackage{amsmath}
\usepackage{graphicx}
\setbeamercolor{bgcolor}{fg=white,bg=orange} 


\setlength{\unitlength}{1.2cm}
\setcounter{tocdepth}{2} 
\setbeamertemplate{footline}
{
	\leavevmode%
	\hbox{%
		\begin{beamercolorbox}[wd=0.4\paperwidth,ht=2.25ex,dp=1ex,center]{title in head/foot}%
			\usebeamerfont{author in head/foot}\today
		\end{beamercolorbox}
		
		\begin{beamercolorbox}[wd=.3\paperwidth,ht=2.25ex,dp=1ex,center]{author in head/foot}%
			\usebeamerfont{title in head/foot} Bailing out Firms Through Banks
		\end{beamercolorbox}
		%
		\begin{beamercolorbox}[wd=.3\paperwidth,ht=2.25ex,dp=1ex,center]{title in head/foot}%
			\usebeamerfont{title in head/foot}
			\insertframenumber{} / \inserttotalframenumber\hspace*{1ex}
		\end{beamercolorbox}}%


	\vskip0pt%
}

\setbeamertemplate{headline}
{
	\leavevmode%
	\hbox{%
		\begin{beamercolorbox}[wd=0.5\paperwidth,ht=2.25ex,dp=1ex,center]{title in head/foot}%
			\usebeamerfont{author in head/foot} \insertsectionhead
		\end{beamercolorbox}
		
		\begin{beamercolorbox}[wd=0.5\paperwidth,ht=2.25ex,dp=1ex,center]{author in head/foot}%
			\usebeamerfont{title in head/foot} \insertsubsectionhead
		\end{beamercolorbox}}
		%


	\vskip0pt%
}

\usepackage{enumitem}
\usepackage{xcolor}
\usepackage{tikz}
\usetikzlibrary{shadows}
\setbeamertemplate{navigation symbols}{}
%\beamertemplatesolidbackgroundcolor{black!5}
%\setbeamercovered{transparent}


\newcommand*{\MyBall}{\tikz \draw [baseline, ball color=red, draw=red] circle (2.5pt);}

%-------------------------------------------------------------




%-------------------------------------------------------------

\begin{document}
	
\title{Bailing out Firms Through Banks}
\author{Chunjie Wang and Martin Waibel (2020)}
\date{\today}

 \renewcommand*\inserttotalframenumber{14	}

\begin{frame}
\maketitle
\end{frame}


\section{Introduction}
\subsection{Motivation}
\begin{frame}
\frametitle{Motivation}

\begin{itemize}[label={\MyBall}]
	\pause
	\item Corporate liquidity management is of crucial importance for firms during the COVID-19 pandemic.

	\pause
	\item Most governments around the world introduced forms of credit relief programs to bridge funding gaps in the corporate sector.

	\pause
	\item Large existing body of literature focuses on optimal government bailout of \textit{financial} institutions.
	
	\pause
	\item Less is known about how non-financial firms should be bailed out.
\end{itemize}

$\implies$ Theory linking federal bailouts with the banking and firm sector.

\end{frame}

\begin{frame}
\frametitle{Relevant Questions}

\begin{itemize}[label={\MyBall}]
	\pause
	\item How can a government bail out firms via the banking sector?
	\pause
	\item At what costs does the bailout come for the government?
	\pause
	\item What changes in case the liquidity shock is private information to the bank and the firm?
	\pause
	\item How should the government decide on whether or not to bail out a firm?
\end{itemize}
\end{frame}

\begin{frame}
\frametitle{Contribution and Results}
\end{frame}

\begin{frame}
\frametitle{Literature}
\textbf{\large\underline{Government bailouts:}}\\ 
Acharya et al. (2014), Blau et al. (2013), Faccio et al. (2006), Gorton and Huang (2004), Diamond and Rajan (2002), Lewrick et al. (2014), Ennis and Keister (2009) \\
\vspace{0.5cm}
\textbf{\large\underline{Corporate liquidity management:}}\\
Holström and Tirole (1997), Tirole (2006), Yun (2009), Campello et al. (2011), Jimenez et al. (2009), Demiroglu and James (2011)


\vspace{1cm}
\textbf{\large\underline{Credit access during Covid-19:}}\\
Banerjee et al. (2020), Bartik et al. (2020), Balduzzi et al. (2020), Dursun-de Neef et al. (2020)


\end{frame}

\section{Model}
\subsection{Model setup}
\begin{frame}
\frametitle{XXX}
\end{frame}

\begin{frame}
\frametitle{XXX}
\end{frame}

\section{Analysis}
\subsection{XXX}
\begin{frame}
\frametitle{XY}
\end{frame}

\begin{frame}
\frametitle{XY}
\end{frame}

\begin{frame}
\frametitle{Extension}

\begin{itemize}[label={\MyBall}]

	\item Idea: In the US banks are forced to keep a minimum of 5\% of the credit risk on their balance sheet. 

	\item Show adjusted table on page 10 

	\item In the current version of the model, whether or not the banks are exposed to part of the credit risk doesn't change their monitoring incentives because the IC and PC need to be fulfilled and the payoff structure cannot be altered. 

	\item Question: How can we move to a more accurate description of reality by allowing for repayment of

\end{itemize}
\end{frame}

\section{Conclusion}
\begin{frame}
\frametitle{Conclusion}
Model of government bailout of firms through financial intermediaries
	\begin{itemize}[label={\MyBall}]
		\item Aggregate liquidity shock exceeds firms' credit lines.\\
		$\implies$ Requires externality-valuing government to make a decision about liquidity injections.
		\pause
		\item Federal liquidity injection via financial intermediaries creates a discontinuous increase in fiscal costs due to moral hazard.
		\pause
		\item Private information about the size of the liquidity shock  aggravates the moral hazard problem locally.
		\pause
		\item Government intervention is costly but not substitutable by the financial intermediaries.
	\end{itemize}

\end{frame}

\section{Appendix}
\begin{frame}
\frametitle{XXX}
\end{frame}




\end{document}