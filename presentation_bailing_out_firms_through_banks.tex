%\documentclass[handout]{beamer} % Activate only if handout slides of presentation desired.
\documentclass[13.8pt]{beamer}

\usetheme{CambridgeUS}

% Packages
\usepackage[utf8]{inputenc}
\usepackage[T1]{fontenc}
\usepackage{amsmath}
\usepackage{graphicx}
\usepackage{eurosym}
\usepackage{hyperref}
\usepackage{amsmath}
\usepackage{graphicx, wrapfig}


\usepackage{tikz}
%\graphicspath{{../../2_Data/output}} % Sets the (relative) path of the figure directory

%%% 3) Tables
\usepackage{booktabs}  % Package optimizing the quality of LaTeX tables
\usepackage{longtable} % Necessary to represent LaTeX tables over multiple pages.
\usepackage{rotating}  % Necessary to represent full-page vertical regression tables 
\usepackage{multirow}  % Necessary to merge cells over multiple columns
\usepackage{makecell}  % Necessary to perform line breaks in a LaTeX table
\usepackage{tabularx}  % Needed to e.g. write long text in cells and get automatic linebreak.
\usepackage{caption}
\usepackage{adjustbox} % needed to adjust the tables to textwidth.
\usepackage{tcolorbox}
\graphicspath{ {./graphs/} }
\setbeamercolor{bgcolor}{fg=white,bg=orange} 


\setlength{\unitlength}{1.2cm}
\setcounter{tocdepth}{2} 
\setbeamertemplate{footline}
{
	\leavevmode%
	\hbox{%
		\begin{beamercolorbox}[wd=0.4\paperwidth,ht=2.25ex,dp=1ex,center]{title in head/foot}%
			\usebeamerfont{author in head/foot}\today
		\end{beamercolorbox}
		
		\begin{beamercolorbox}[wd=.3\paperwidth,ht=2.25ex,dp=1ex,center]{author in head/foot}%
			\usebeamerfont{title in head/foot} Bailing out Firms Through Banks
		\end{beamercolorbox}
		%
		\begin{beamercolorbox}[wd=.3\paperwidth,ht=2.25ex,dp=1ex,center]{title in head/foot}%
			\usebeamerfont{title in head/foot}
			\insertframenumber{} / \inserttotalframenumber\hspace*{1ex}
		\end{beamercolorbox}}%


	\vskip0pt%
}

\setbeamertemplate{headline}
{
	\leavevmode%
	\hbox{%
		\begin{beamercolorbox}[wd=0.5\paperwidth,ht=2.25ex,dp=1ex,center]{title in head/foot}%
			\usebeamerfont{author in head/foot} \insertsectionhead
		\end{beamercolorbox}
		
		\begin{beamercolorbox}[wd=0.5\paperwidth,ht=2.25ex,dp=1ex,center]{author in head/foot}%
			\usebeamerfont{title in head/foot} \insertsubsectionhead
		\end{beamercolorbox}}
		%


	\vskip0pt%
}

\usepackage{enumitem}
\usepackage{xcolor}
\usepackage{tikz}
\usetikzlibrary{shadows}
\setbeamertemplate{navigation symbols}{}
%\beamertemplatesolidbackgroundcolor{black!5}
%\setbeamercovered{transparent}


\newcommand*{\MyBall}{\tikz \draw [baseline, ball color=red, draw=red] circle (2.5pt);}


\begin{document}
	
\title{Bailing out Firms Through Banks Jan is the best}
\author{Chunjie Wang and Martin Waibel (2020)}
\date{\today}

 \renewcommand*\inserttotalframenumber{14	}

\begin{frame}
\maketitle
\end{frame}



\section{Model}
\subsection{Model setup}

\begin{frame}
\frametitle{Basic Setup: Contracted Liquidity Shock (Tirole, 2006)}
\begin{itemize}[label={\MyBall}]
\item competitive credit market; effort of entrepreneur is not contractible $\implies$ \textbf{incentive compatibility};  \textbf{participation constraint}
\item liquidity shock $\rho$ randomly drawn from commonly known cdf $F(\rho)$
\item The contract at $t=0$ also states the maximum liquidity shock $\rho ^*$ at $t=1$ that the investor agrees to finance.
\item Pledgeable in come to bank in a competitive credit market is thus \begin{align*}
    \mathcal{P}(\rho ^*)=F(\rho ^*)p_H\left( R-\frac{B}{\Delta p}\right)-\int ^{\rho ^*}_0 \rho f(\rho)d\rho=I-A
\end{align*}
\item The expected payoff to the firm is \begin{equation*}
    NPV_{firm}(\rho ^*)=U_{firm}(\rho ^*)=F(\rho ^*)p_HR-\left( I+\int ^{\rho ^*}_0 \rho f(\rho)d\rho \right)
\end{equation*}
\item \begin{equation*}
 \rho ^* \in [p_H(R-\frac{B}{\Delta p}),  p_HR]\implies \text{Secure a line of credit of } \rho^*
\end{equation*}

\end{itemize}

\end{frame}



\begin{frame}
\frametitle{Government Intervention and New Moral Hazard}
\begin{itemize}[label={\MyBall}]
\item An aggregate liquidity shock $\rho>\rho^*$ hits all firms: government with deep pocket intervenes at $t=1$.
\item The government can only monitor the firm through the investor.
\item The firm at $t=1$, if not monitored, chooses between:
\begin{itemize}[label={\MyBall}]
\item \textbf{reinvest:} invest government liquidity together with the credit line contracted with the investor;
\item \textbf{default:} consume the government injection.

\end{itemize}


\begin{table}[H]
\centering
\caption{Cashflows Upon Federal Liquidity Injection}
\resizebox{2in}{!}{\begin{tabular}{lcccc}
\hline
Decision &  Agent            & $t=0$      & $t=1$     & $t=2$  \\ \hline
\multirow{2}{*}{reinvest} & Firm & $-A$     &   0    & $p_H\frac{B}{\Delta p}$ \\
& Bank     & $-(I-A)$ & $-\rho$ * & $p_H\left( R-\frac{B}{\Delta p}\right)$ \\ \hline
\hline
%t            & $0$      & $1$     & $2$  \\ \hline
\multirow{2}{*}{default}& Firm & $-A$     & $+(\rho-\rho ^*)$       & 0 \\
& Bank     & $-(I-A)$ & 0  & 0 \\ \hline
\end{tabular}}
\end{table}

\item $ \rho ^* \in [p_H(R-\frac{B}{\Delta p}),  p_HR] \implies$ \textbf{investor does not have incentive to monitor regardless of $\rho$}, he would rather the project go south.
\end{itemize}
\end{frame}


\begin{frame}
\frametitle{Fiscal Cost of Intervention}
When $\rho \leq \rho ^*+\frac{p_HB}{\Delta p}$, the firm reinvests; otherwise the firm defaults.

\begin{wrapfigure}{r}{8cm}
\includegraphics[scale=0.34]{utility1}
\end{wrapfigure} 

The government has to incentivize the investor to overcome the problem by paying him $D$ to monitor the firm, a discontinuity increase in fiscal cost:
\begin{align*}
D=\rho ^*-p_H(R-\frac{B}{\Delta p})  
\end{align*}
\par
\textbf{Policy Recommendation I:} \textit{The government decides to inject liquidity in response to a liquidity shock of size $\rho$ whenever $\phi-g(\rho)>0 \iff \rho<g^{-1}(\phi)$ assuming that the government is able to observe the liquidity shock.}
\end{frame}


\subsection{Over-Report Tendency}


\begin{frame}
\frametitle{Over-Report Tendency}
\begin{itemize}[label={\MyBall}]

\item It is particularly likely that the true liquidity shock $\rho_t$ is private information to the firm and the investor.
\item This informational asymmetry allows the firm-investor pair to over-report the actual liquidity shock as $\rho_{r} = \rho_{t}+\sigma$ for some $\sigma \geq 0$ so as to jointly realize extra payoff.
\item The problem is aggravated around the cutoff $\rho_{cf}$. This is because  over-reporting for some $\rho_t$ slightly below $\rho_{cf}$ could lead to a discontinuous increase in fiscal costs.
\end{itemize}
\begin{center}
\includegraphics[scale=0.3]{utility2}
\end{center}

\end{frame}

\begin{frame}
\frametitle{Over-Report Tendency}
\begin{itemize}[label={\MyBall}]
\item Our goal: to find policy cutoff $\tilde{\rho}$ that causes the least loss to the government compared to a perfect world without over-report tendency.

\item Observing $\rho_r$, the government forms an estimate of $\sigma=\rho_r-\rho_t$, denoted as $\psi$, which is continuously distributed with cdf $H(\psi)$.
\item If the government grants $\rho_r-\rho^*+D$, the expected loss (compared to perfect world) is $\int h(\psi)\psi \,\,d\psi+\mathbb{P}(\rho_r-\psi \leq \rho_{cf})D$, denoted as \textcolor{red}{\textit{type I loss}}; if the government only grants $\rho_r-\rho^*$, the expected loss is $\int h(\psi)\psi \,\,d\psi+\mathbb{P}(\rho_r-\psi > \rho_{cf})(\phi-D)$, denoted as \textcolor{red}{\textit{type II loss}}. 
\item By definition of $\tilde{\rho}$, it solves when the two types loss equalize.

\begin{align*}
         \tilde{\rho}&=\rho_{cf}+H^{-1}\left( \frac{D}{\phi} \right).
\end{align*}
\item \textbf{Policy Recommendation 2:} \textit{In case of informational asymmetries of actual liquidity shock $\rho_t$ the fiscal authority should set $\tilde{\rho}=\rho_{cf}+H^{-1}\left( \frac{D}{\phi} \right)$.}
\end{itemize}
\end{frame}

\section{Appendix}

\subsection{Model setup}
\begin{frame}
\frametitle{Basic Setup: Contracted Liquidity Shock (Tirole, 2006)}
\begin{itemize}[label={\MyBall}]
	\item $t\in \{ 0,1,2\}$
	\item Two agents: firm (endowment $A$) and investor
	\item One project: 
		\begin{itemize}
		\item $t=0$: upfront investment $I$;
		\item $t=1$: liquidity shock $\rho$ randomly drawn from cdf $F(\rho)$;
		\item $t=2$: the project yields
							$0$ if $\rho$ is not committed;
							$p_H R$ if $\rho$ is committed and firm exerts effort;
							$p_L R$ if $\rho$ is committed and firm shirks.							
	\end{itemize}
	\item Moral hazard: effort is not contractible and shirking yields private benefit $B$ to the firm.  
	
\end{itemize}

\end{frame}

\begin{frame}
\frametitle{Basic Setup: Contracted Liquidity Shock (Tirole, 2006)}
\begin{itemize}[label={\MyBall}]
\item The financing contract at $t=0$ needs to be incentive compatible: the firm requires $p_H\frac{B}{\Delta p}$ to exert effort.
\item The contract at $t=0$ also states the maximum liquidity shock $\rho ^*$ at $t=1$ that the investor agrees to finance.
\item Pledgeable in come to bank in a competitive credit market is thus \begin{align*}
    \mathcal{P}(\rho ^*)=F(\rho ^*)p_H\left( R-\frac{B}{\Delta p}\right)-\int ^{\rho ^*}_0 \rho f(\rho)d\rho=I-A
\end{align*}
\item The expected payoff to the firm is \begin{equation*}
    NPV_{firm}(\rho ^*)=U_{firm}(\rho ^*)=F(\rho ^*)p_HR-\left( I+\int ^{\rho ^*}_0 \rho f(\rho)d\rho \right)
\end{equation*}

\end{itemize}

\end{frame}

\begin{frame}
\frametitle{Basic Setup: Contracted Liquidity Shock (Tirole, 2006)}
\begin{itemize}[label={\MyBall}]
\item $U_{firm}(\rho ^*)$ peaks at $\rho *=p_HR$ and $\mathcal{P}(\rho ^*)$ peaks at $\rho *=p_H(R-\frac{B}{\Delta p})$.
\includegraphics[scale=0.4]{Tirole}

\item if $\mathcal{P}(p_HR)\geq I-A$: $\rho^*$ is set to equal $p_HR$;
\item if $\mathcal{P}(p_HR)< I-A\leq  \mathcal{P}\left(p_H\left(R-\frac{B}{\Delta p} \right) \right)$: $\rho ^* \in \left[p_H(R-\frac{B}{\Delta p}), p_HR \right[$ solves $\mathcal{P}(\rho ^*)=I-A$


\end{itemize}

\end{frame}


\begin{frame}
\frametitle{Basic Setup: Contracted Liquidity Shock (Tirole, 2006)}
\begin{itemize}[label={\MyBall}]
\item $ \rho ^* \in [p_H(R-\frac{B}{\Delta p}),  p_HR]$ implies the firm needs to plan its liquidity management at $t=0$ according to either of the following specifications or any combination of them:
\vspace{0.5cm}
\begin{itemize}[label={\MyBall}]
\item \textbf{secure a line of credit of $\rho^*$ with no right to dilute the existing share by issuing new claims at $t=1$;}
    \item secure a credit line of $\rho^*-p_H(R-\frac{B}{\Delta p})$ with a right to dilute the existing share at $t=1$;
\end{itemize}
\vspace{0.5cm}
\item It turns out which specification the firm decides to follow does not matter in our model, thus throughout the paper, we assume the first case. 


\end{itemize}
\end{frame}

\end{document}