%\documentclass[handout]{beamer} % Activate only if handout slides of presentation desired.
\documentclass{beamer}

\usetheme{CambridgeUS}

% Packages
\usepackage[latin1]{inputenc}
\usepackage[T1]{fontenc}
\usepackage{amsmath}
\usepackage{graphicx}
\usepackage{eurosym}
\usepackage{hyperref}
\usepackage{amsmath}
\usepackage{graphicx}

\setlength{\unitlength}{1.2cm}
\setcounter{tocdepth}{2} 
\setbeamertemplate{footline}
{
	\leavevmode%
	\hbox{%
		\begin{beamercolorbox}[wd=0.4\paperwidth,ht=2.25ex,dp=1ex,center]{author in head/foot}%
			\usebeamerfont{author in head/foot}\today
		\end{beamercolorbox}
		
		\begin{beamercolorbox}[wd=.3\paperwidth,ht=2.25ex,dp=1ex,center]{author in head/foot}%
			\usebeamerfont{title in head/foot} Martin Waibel
		\end{beamercolorbox}
		%
		\begin{beamercolorbox}[wd=.3\paperwidth,ht=2.25ex,dp=1ex,center]{title in head/foot}%
			\usebeamerfont{title in head/foot}
			\insertframenumber{} / \inserttotalframenumber\hspace*{1ex}
		\end{beamercolorbox}}%


	\vskip0pt%
}

\setbeamertemplate{navigation symbols}{}
%\beamertemplatesolidbackgroundcolor{black!5}
%\setbeamercovered{transparent}




%-------------------------------------------------------------




%-------------------------------------------------------------

\begin{document}
	
\title{Bailing out Firms Through Banks}
\author{Chnujie Wang and Martin Waibel (2020)}
\date{\today}

 \renewcommand*\inserttotalframenumber{14	}

\begin{frame}
\maketitle
\end{frame}


\begin{frame}
\frametitle{Introduction}
\end{frame}

\begin{frame}
\frametitle{Conclusion}
\end{frame}

\begin{frame}
\frametitle{Extension}
\begin{itemize}
	\item Idea: In the US banks are forced to keep a minimum of 5\% of the credit risk on their balance sheet. 

	\item Show adjusted table on page 10 

	\item In the current version of the model, whether or not the banks are exposed to part of the credit risk doesn't change their monitoring incentives because the IC and PC need to be fulfilled and the payoff structure cannot be altered. 

	\item Question: How can we move to a more accurate description of reality by allowing for repayment of government liquidity injection and thus allow for incentivising banks via an exposure to credit risk ?

	% Comment to Jan: Could this be a test for how valid Tirole's assumption actually is. In particular, if we were to find emirically that after the Covid crisis German and American banks (with different credit risk exposure) had the same degree of funding to firms.

	\item TEST fdgfdfgdgfd asdgads
\end{itemize}

\end{frame}


\end{document}